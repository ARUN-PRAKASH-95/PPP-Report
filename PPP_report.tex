\documentclass[a4paper,12pt]{article}
    \usepackage{amsmath}
    \usepackage{graphicx}
    \usepackage{cite}
	\usepackage{caption}
	\usepackage{subcaption}
    \usepackage{multirow}
    \usepackage[top=1in, bottom=1.1in, left=1.1in, right=1.1in]{geometry}
    \usepackage{fancyhdr}
    \usepackage{relsize}
	\makeatletter
    \setlength{\@fptop}{0pt}
    \makeatother

\title{\Huge{\textbf{Static Analysis of Beam structures using Refined 1D beam model}}\\[0.5em]\smaller \textbf{Personal Programming Project Report  Winter semester 2019/2020}}
\author{\LARGE{Arun Prakash Ganapathy}\\
\\
\Large{Mat.Nr. 63876}\\
\\
\Large{E-Mail: arun-prakash.ganapathy@student.tu-freiberg.de}
\\
\\
\Large{Supervised by: Mr.Jeffy Abraham}
}
\date{}

\begin{document}
\maketitle
\newpage
\vspace*{50px}
\section{Introduction}
\indent\indent\indent\indent Beam structures are widely used in many engineering applications such as aircraft wings, helicopter blades, concrete beams in constructions etc., Classical 1-D models for beams made of isotropic materials are based on Euler-Bernoulli beam and Timoshenko theories. They yield better results for slender beams than short beams. If the cross-section of a bar is considered rigid, the problem can be considered a 1D one. In other words, the value of the displacement on the axis is enough to describe the deformation. But the classical 1D beam models does not take into the account of the non-classical effects like transverse shear deformation, torsion etc.,
\begin{figure}[htbp]
\begin{center}
\includegraphics[width=0.8\textwidth]{{1.png}}
 \caption{Refined 1D element based on CUF}
 \label{fig:Refined 1D element}
\end{center}
\end{figure}\\
\indent\indent In order to overcome this drawback, the displacement values, should not be considered constant over the cross-section. Rather the displacement field that was originally defined in a 1D domain now becomes a 3D field. This can be done using Refined 1D beam models based on Carrera Unified Formulation(CUF). Refined 1D models use polynomial expansions(like Taylor,Lagrange polynomials etc.,) to approximate the displacement values across the cross section of the beam.Therefore this theory requires two major steps for FEM implementation. 1.FE discretization on the axis of the beam element (1D) 2.Expansion across the cross section(2D) using polynomials which gives the displacement field. Refined theories are necessary to cope with unconventional
cross-section geometries, short beams, orthotropic materials and non-homogenous sections. The figure \ref{fig:Refined 1D element} shows the axial approximation using 1D shape functions and cross-sectional approximation using polynomials

\newpage
\section{1D Refined models with Lagrange Expansion class}
\indent\indent\indent\indent In a displacement based approach different class of functions are used to describe the displacment field of cross section like harmonics,polynomials,exponentials etc., In this project Lagrange polynomial expansions is used to approximate the displacement field of the cross section. The Lagrange Expansion(LE) 1D models have the folowing main characters 
\\
1. LE model variables and BCs can be located above the physical surfaces of the structure\\ 
2. The unknown variables of the problems are only displacements of the nodes. No rotation or higher order variables are used to describe the displacement field.\\
3. Locally refined models can be easily built since Lagrange polynomial sets can be arbitrarily
spread above the cross-section.\\
\indent\indent In this method the expansion functions(F$_\tau$) coincide with the Lagrange polynomials. Since Lagrange polynomials are usually given in normalized or natural coordinates Isoparametric formulation can be exploited. The simplest Lagrange polynomial is the four-point(L4) set which is shown in the figure \ref{fig:L4 element} and the polynomials are given by the equation(2.1) 
\begin{figure}[htbp]
\begin{center}
\includegraphics[width=0.8\textwidth]{{2.png}}
 \caption{Four node Lagrange element(L4) in normalized and actual geometry}
 \label{fig:L4 element}
\end{center}
\end{figure}

$$F_\tau = \frac{1}{4}(1+\alpha\alpha_\tau)(1+\beta\beta_\tau), \tau = 1,2,3,4 \hspace{1in}(2.1)$$ 
where $\alpha$ and $\beta$ are the normalized coordinates and $\alpha_\tau$ and $\beta_\tau$ are the coordinates of the four nodes which is given in the table \ref{tab:table1}.In the same way triangular elements(L3) and  higher order elements like (L9) or L(16) can be created using their respective polynomials.
\begin{table}[h!]
  \begin{center}
     \begin{tabular}{c|r|r} 
      \textbf{Point} & \textbf{$\alpha_\tau$} & \textbf{$\beta_\tau$}\\
      \hline
      1 & -1 &  1\\
      2 &  1 & -1\\
      3 &  1 &  1\\
      4 & -1 &  1 
    \end{tabular}
    \caption{Normalized coordinates of the four     points of an L4 element}
    \label{tab:table1}
  \end{center}
\end{table}\\

\newpage
\subsection{Isoparametic Formulation}

\end{document}